% Options for packages loaded elsewhere
\PassOptionsToPackage{unicode}{hyperref}
\PassOptionsToPackage{hyphens}{url}
%
\documentclass[
]{article}
\usepackage{lmodern}
\usepackage{amssymb,amsmath}
\usepackage{ifxetex,ifluatex}
\ifnum 0\ifxetex 1\fi\ifluatex 1\fi=0 % if pdftex
  \usepackage[T1]{fontenc}
  \usepackage[utf8]{inputenc}
  \usepackage{textcomp} % provide euro and other symbols
\else % if luatex or xetex
  \usepackage{unicode-math}
  \defaultfontfeatures{Scale=MatchLowercase}
  \defaultfontfeatures[\rmfamily]{Ligatures=TeX,Scale=1}
\fi
% Use upquote if available, for straight quotes in verbatim environments
\IfFileExists{upquote.sty}{\usepackage{upquote}}{}
\IfFileExists{microtype.sty}{% use microtype if available
  \usepackage[]{microtype}
  \UseMicrotypeSet[protrusion]{basicmath} % disable protrusion for tt fonts
}{}
\makeatletter
\@ifundefined{KOMAClassName}{% if non-KOMA class
  \IfFileExists{parskip.sty}{%
    \usepackage{parskip}
  }{% else
    \setlength{\parindent}{0pt}
    \setlength{\parskip}{6pt plus 2pt minus 1pt}}
}{% if KOMA class
  \KOMAoptions{parskip=half}}
\makeatother
\usepackage{xcolor}
\IfFileExists{xurl.sty}{\usepackage{xurl}}{} % add URL line breaks if available
\IfFileExists{bookmark.sty}{\usepackage{bookmark}}{\usepackage{hyperref}}
\hypersetup{
  pdftitle={Chapter 3: Covariance, regression, and correlation},
  hidelinks,
  pdfcreator={LaTeX via pandoc}}
\urlstyle{same} % disable monospaced font for URLs
\usepackage[margin=1in]{geometry}
\usepackage{graphicx}
\makeatletter
\def\maxwidth{\ifdim\Gin@nat@width>\linewidth\linewidth\else\Gin@nat@width\fi}
\def\maxheight{\ifdim\Gin@nat@height>\textheight\textheight\else\Gin@nat@height\fi}
\makeatother
% Scale images if necessary, so that they will not overflow the page
% margins by default, and it is still possible to overwrite the defaults
% using explicit options in \includegraphics[width, height, ...]{}
\setkeys{Gin}{width=\maxwidth,height=\maxheight,keepaspectratio}
% Set default figure placement to htbp
\makeatletter
\def\fps@figure{htbp}
\makeatother
\setlength{\emergencystretch}{3em} % prevent overfull lines
\providecommand{\tightlist}{%
  \setlength{\itemsep}{0pt}\setlength{\parskip}{0pt}}
\setcounter{secnumdepth}{-\maxdimen} % remove section numbering

\title{Chapter 3: Covariance, regression, and correlation}
\author{}
\date{\vspace{-2.5em}}

\begin{document}
\maketitle

\hypertarget{covariance}{%
\subsection{Covariance}\label{covariance}}

Covariance is a measure of association and the covariance between x and
y would be denoted by \(\sigma(x, y)\). If x and y are independent then
\(\sigma(x, y) = 0\), BUT if \(\sigma(x, y) = 0\), x and y aren't
necessarily independent.

\hypertarget{useful-identities-for-cov}{%
\subsubsection{Useful identities for
cov}\label{useful-identities-for-cov}}

Covariance of x with itself = variance of x:
\[\sigma(x, x) = \sigma^2(x)\]

For constants (here represented by a): \[\sigma(a, x) = 0\]
\[\sigma(ax, y) = a\sigma(x, y)\] \[\sigma^2(a, x) = a^2\sigma^2(x)\]
\[\sigma[(a + x), y] = \sigma(x, y)\]

The covariance of 2 sums can be written as the sum of covariances,
i.e.~just multiply out the brackets (I've left this blank, do it
yourself or check book): \[\sigma[(x + y),(w + z)] = ...\]

Variance of a sum is sum of variances and covariances (figure this out):
\[\sigma^2(x + y) = ...\]

\hypertarget{least-squares-linear-regression}{%
\subsection{Least squares linear
regression}\label{least-squares-linear-regression}}

Linear model: \[y = \alpha + \beta{x} + e\]

Continuing on, \(\alpha\) and \(\beta\) will be the true population
values and a and b will be the intercept and slope for the line of best
fit derived from observed data. The derivation of a and b using the
least-squares model can be found on pages 39-41. Buuut, who cares about
that, here are the results: \[a = \bar{y} - b\bar{x}\]
\[b = \frac{Cov(x, y)} {Var(x)}\]

\hypertarget{properties-of-least-squares}{%
\subsubsection{Properties of least
squares}\label{properties-of-least-squares}}

6 in the book, just writing down important/not obvious ones.

\begin{itemize}
\tightlist
\item
  The mean residual (\(\bar{e}\)) is 0
\item
  Residual errors are uncorrelated with predictor variable x (see book
  for why)

  \begin{itemize}
  \tightlist
  \item
    BUT e and x may not be independent if the relationship between x and
    y is non-linear. If it is truly non-linear \(E(e|x) != 0\)
  \end{itemize}
\item
  Variance of e can vary with x, in this situation the the regression is
  said to display heteroscedasticity (see Figure 3.4 for great
  illustration)
\item
  The regression of y on x is different to the regression of x on y!
\end{itemize}

\hypertarget{correlation}{%
\subsection{Correlation}\label{correlation}}

Correlation coefficient between x and y:
\[r(x, y) = \frac{Cov(x, y)} {\sqrt{Var(x) Var(y)}}\]

The correlation coefficient is a dimensionless measure of association
and it is symmetrical (i.e.~\(r(x, y) = r(y, x)\)).

Scaling x or y by constants does not change the correlation coefficient,
but it does affect variances and covariances.

The correlation coefficient is a standardised regression coefficient
-\textgreater{} the regression coefficient resulting from rescaling x
and y such that each has unit variance).

\(r^2\) assumes \(E(y|x)\) is linear!

\hypertarget{differential-selection-brief-intro}{%
\subsection{Differential selection (brief
intro)}\label{differential-selection-brief-intro}}

The directional selection differential, \(S\), is the difference between
the mean phenotype within that generation before selection (\(\mu_s\))
and the mean phenotype within that generation after (\(\mu\)) selection.
\[S = \mu_s - \mu\]

If all individuals have equal fertility and viability then selecting
individuals won't change anything so \(\mu_s = \mu\) and \(S = 0\).

If \(W(z)\) is the probability that individuals with phenotype \(z\)
survive to reproduce and \(p(z)\) is the density of \(z\) (pretty much
means distribution) before selection, then the density after selection
is: \[p_{s}(z) = \frac{W(z)p(z)} {\int W(z)p(z)dz}\]

The denominator here is the mean individual fitness (\(\bar{W}\)). The
relative fitness of \(z\) is \(w(z) = \frac{W(z)} {\bar{W}}\).

After some sweet derivation (see page 46), you finish with:
\[S = \sigma[z, w(z)]\]

Therefore the directional selection is equivalent to the covariance of
the phenotype and the relative fitness.

If you regress offspring phenotype on the midparent phenotype and that
relationship is linear with slope \(\beta\), a change in mean midparent
phenotype induces an expected change in mean phenotype across
generations equal to:
\[\Delta\mu = \mu_0 - \mu = \beta(\mu_s - \mu) = \beta{S}\]

This is the breeders' equation!

\hypertarget{correlation-between-genotype-and-phenotype-brief-intro}{%
\subsection{Correlation between genotype and phenotype (brief
intro)}\label{correlation-between-genotype-and-phenotype-brief-intro}}

Only when there is no gene-environment interaction is the variance
explained by genetics (broad-sense heritability) the equation below:
\[H^2 = \frac{\sigma^2_G} {\sigma^2_z}\],

where \(z\) is the phenotype and G is the sum of the total effects (not
just additive) at all loci on the trait.

The slope of a midparent-offspring regression provides an estimate of
the proportion of the phenotypic variance that is attributable to
additive genetic factors (the narrow-sense heritability).
\[h^2 = \frac{\sigma^2_A} {\sigma^2_z}\]

So as \(h^2\) is just the regression of offspring phenotype on midparent
phenotype it can actually be used in the breeders' equation!
\[\Delta\mu = h^2S\]

So the narrow-sense heritability can be thought of as the efficiency of
the response to selection. If \(h^2 = 0\) there can be no evolutionary
change regardless of strength of selection. Although this should be
obvious because if \(h^2\) is 0 then there is clearly no passing of
genetic material onto the next generation that is influencing that
trait.

\end{document}

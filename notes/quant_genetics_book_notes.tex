% Options for packages loaded elsewhere
\PassOptionsToPackage{unicode}{hyperref}
\PassOptionsToPackage{hyphens}{url}
%
\documentclass[
]{book}
\usepackage{lmodern}
\usepackage{amssymb,amsmath}
\usepackage{ifxetex,ifluatex}
\ifnum 0\ifxetex 1\fi\ifluatex 1\fi=0 % if pdftex
  \usepackage[T1]{fontenc}
  \usepackage[utf8]{inputenc}
  \usepackage{textcomp} % provide euro and other symbols
\else % if luatex or xetex
  \usepackage{unicode-math}
  \defaultfontfeatures{Scale=MatchLowercase}
  \defaultfontfeatures[\rmfamily]{Ligatures=TeX,Scale=1}
\fi
% Use upquote if available, for straight quotes in verbatim environments
\IfFileExists{upquote.sty}{\usepackage{upquote}}{}
\IfFileExists{microtype.sty}{% use microtype if available
  \usepackage[]{microtype}
  \UseMicrotypeSet[protrusion]{basicmath} % disable protrusion for tt fonts
}{}
\makeatletter
\@ifundefined{KOMAClassName}{% if non-KOMA class
  \IfFileExists{parskip.sty}{%
    \usepackage{parskip}
  }{% else
    \setlength{\parindent}{0pt}
    \setlength{\parskip}{6pt plus 2pt minus 1pt}}
}{% if KOMA class
  \KOMAoptions{parskip=half}}
\makeatother
\usepackage{xcolor}
\IfFileExists{xurl.sty}{\usepackage{xurl}}{} % add URL line breaks if available
\IfFileExists{bookmark.sty}{\usepackage{bookmark}}{\usepackage{hyperref}}
\hypersetup{
  pdftitle={Notes for: Walsh and Lynch. Genetics and Analysis of Quantitative Traits},
  pdfauthor={Thomas Battram},
  hidelinks,
  pdfcreator={LaTeX via pandoc}}
\urlstyle{same} % disable monospaced font for URLs
\usepackage{longtable,booktabs}
% Correct order of tables after \paragraph or \subparagraph
\usepackage{etoolbox}
\makeatletter
\patchcmd\longtable{\par}{\if@noskipsec\mbox{}\fi\par}{}{}
\makeatother
% Allow footnotes in longtable head/foot
\IfFileExists{footnotehyper.sty}{\usepackage{footnotehyper}}{\usepackage{footnote}}
\makesavenoteenv{longtable}
\usepackage{graphicx}
\makeatletter
\def\maxwidth{\ifdim\Gin@nat@width>\linewidth\linewidth\else\Gin@nat@width\fi}
\def\maxheight{\ifdim\Gin@nat@height>\textheight\textheight\else\Gin@nat@height\fi}
\makeatother
% Scale images if necessary, so that they will not overflow the page
% margins by default, and it is still possible to overwrite the defaults
% using explicit options in \includegraphics[width, height, ...]{}
\setkeys{Gin}{width=\maxwidth,height=\maxheight,keepaspectratio}
% Set default figure placement to htbp
\makeatletter
\def\fps@figure{htbp}
\makeatother
\setlength{\emergencystretch}{3em} % prevent overfull lines
\providecommand{\tightlist}{%
  \setlength{\itemsep}{0pt}\setlength{\parskip}{0pt}}
\setcounter{secnumdepth}{5}

\title{Notes for: Walsh and Lynch. Genetics and Analysis of Quantitative Traits}
\author{Thomas Battram}
\date{2020-06-11}

\begin{document}
\maketitle

{
\setcounter{tocdepth}{1}
\tableofcontents
}
\hypertarget{preface}{%
\chapter*{Preface}\label{preface}}
\addcontentsline{toc}{chapter}{Preface}

This is a good book, but if I make it through the whole thing I deserve several medals and some cake.

\hypertarget{an-overview-of-quantitative-genetics}{%
\chapter{An overview of quantitative genetics}\label{an-overview-of-quantitative-genetics}}

BORWANG!!!

This chapter just introduces the book and some simple concepts.

\hypertarget{properties-of-distributions}{%
\chapter{Properties of distributions}\label{properties-of-distributions}}

ALSO BORWANG!

You can guess what this chapter was on and also how much of a hoot it was\ldots{}

\hypertarget{covariance-regression-and-correlation}{%
\chapter{Covariance, regression, and correlation}\label{covariance-regression-and-correlation}}

Placeholder

\hypertarget{covariance}{%
\section{Covariance}\label{covariance}}

\hypertarget{useful-identities-for-covariance}{%
\subsection{Useful identities for covariance}\label{useful-identities-for-covariance}}

\hypertarget{least-squares-linear-regression}{%
\section{Least squares linear regression}\label{least-squares-linear-regression}}

\hypertarget{properties-of-least-squares}{%
\subsection{Properties of least squares}\label{properties-of-least-squares}}

\hypertarget{correlation}{%
\section{Correlation}\label{correlation}}

\hypertarget{differential-selection-brief-intro}{%
\section{Differential selection (brief intro)}\label{differential-selection-brief-intro}}

\hypertarget{correlation-between-genotype-and-phenotype-brief-intro}{%
\section{Correlation between genotype and phenotype (brief intro)}\label{correlation-between-genotype-and-phenotype-brief-intro}}

\hypertarget{end-of-chapter-questions}{%
\section{End of chapter questions}\label{end-of-chapter-questions}}

\hypertarget{properties-of-single-loci}{%
\chapter{Properties of single loci}\label{properties-of-single-loci}}

Placeholder

\hypertarget{introduction}{%
\section{Introduction}\label{introduction}}

\hypertarget{allele-and-genotype-frequencies}{%
\section{Allele and genotype frequencies}\label{allele-and-genotype-frequencies}}

\hypertarget{the-transmission-of-genetic-information}{%
\section{The transmission of genetic information}\label{the-transmission-of-genetic-information}}

\hypertarget{the-hardy-weinberg-principle}{%
\subsection{The Hardy-Weinberg principle}\label{the-hardy-weinberg-principle}}

\hypertarget{sex-linked-loci}{%
\subsection{Sex-linked loci}\label{sex-linked-loci}}

\hypertarget{polyploidy}{%
\subsection{Polyploidy}\label{polyploidy}}

\hypertarget{age-structure}{%
\subsection{Age structure}\label{age-structure}}

\hypertarget{testing-for-hardy-weinberg-proportions}{%
\subsection{Testing for Hardy-Weinberg proportions}\label{testing-for-hardy-weinberg-proportions}}

\hypertarget{characterising-the-influence-of-a-locus-on-the-phenotype}{%
\section{Characterising the influence of a locus on the phenotype}\label{characterising-the-influence-of-a-locus-on-the-phenotype}}

\hypertarget{the-basis-of-dominance}{%
\section{The basis of dominance}\label{the-basis-of-dominance}}

\hypertarget{fishers-decomposition-of-the-genotypic-value}{%
\section{Fisher's decomposition of the genotypic value}\label{fishers-decomposition-of-the-genotypic-value}}

\hypertarget{partioning-the-genetic-variance.}{%
\section{Partioning the genetic variance.}\label{partioning-the-genetic-variance.}}

\hypertarget{additive-effects-average-excesses-and-breeding-values}{%
\section{Additive effects, average excesses and breeding values}\label{additive-effects-average-excesses-and-breeding-values}}

\hypertarget{extensions-for-multiple-alleles-and-non-random-mating}{%
\section{Extensions for multiple alleles and non random mating}\label{extensions-for-multiple-alleles-and-non-random-mating}}

\hypertarget{average-excess}{%
\subsection{Average excess}\label{average-excess}}

\hypertarget{additive-effects}{%
\subsection{Additive effects}\label{additive-effects}}

\hypertarget{additive-genetic-variance}{%
\subsection{Additive genetic variance}\label{additive-genetic-variance}}

\hypertarget{end-of-chapter-questions-1}{%
\section{End of chapter questions}\label{end-of-chapter-questions-1}}

\hypertarget{sources-of-genetic-variation-for-multilocus-traits}{%
\chapter{Sources of genetic variation for multilocus traits}\label{sources-of-genetic-variation-for-multilocus-traits}}

Placeholder

\hypertarget{epistasis}{%
\section{Epistasis}\label{epistasis}}

\hypertarget{a-general-least-squares-model-for-genetic-effects}{%
\section{A general least-squares model for genetic effects}\label{a-general-least-squares-model-for-genetic-effects}}

\hypertarget{extension-to-haploids-and-polyploids}{%
\subsection{Extension to haploids and polyploids}\label{extension-to-haploids-and-polyploids}}

\hypertarget{linkage}{%
\section{Linkage}\label{linkage}}

\hypertarget{effect-of-disequilibrium-of-the-genetic-variance}{%
\section{Effect of disequilibrium of the genetic variance}\label{effect-of-disequilibrium-of-the-genetic-variance}}

\hypertarget{the-evidence}{%
\subsection{The evidence}\label{the-evidence}}

\hypertarget{end-of-chapter-questions-2}{%
\section{End of chapter questions}\label{end-of-chapter-questions-2}}

\hypertarget{sources-of-environmental-variation}{%
\chapter{Sources of Environmental Variation}\label{sources-of-environmental-variation}}

Placeholder

\hypertarget{extension-of-the-linear-model-to-phenotypes}{%
\section{Extension of the linear model to phenotypes}\label{extension-of-the-linear-model-to-phenotypes}}

\hypertarget{special-environmental-effects}{%
\section{Special environmental effects}\label{special-environmental-effects}}

\hypertarget{within-individual-variation}{%
\subsection{Within-individual variation}\label{within-individual-variation}}

\hypertarget{developmental-homeostasis-and-homozygosity}{%
\subsection{Developmental homeostasis and homozygosity}\label{developmental-homeostasis-and-homozygosity}}

\hypertarget{repeatability}{%
\subsection{Repeatability}\label{repeatability}}

\hypertarget{general-environmental-effects-of-maternal-origin}{%
\section{General environmental effects of maternal origin}\label{general-environmental-effects-of-maternal-origin}}

\hypertarget{genotype-x-environment-interaction}{%
\section{Genotype x environment interaction}\label{genotype-x-environment-interaction}}

\hypertarget{resemblance-between-relatives}{%
\chapter{Resemblance between relatives}\label{resemblance-between-relatives}}

Placeholder

\hypertarget{measures-of-relatedness}{%
\section{Measures of relatedness}\label{measures-of-relatedness}}

\hypertarget{coefficients-of-identity}{%
\subsection{Coefficients of identity}\label{coefficients-of-identity}}

\hypertarget{coefficients-of-coancestry-and-inbreeding}{%
\subsection{Coefficients of coancestry and inbreeding}\label{coefficients-of-coancestry-and-inbreeding}}

\hypertarget{the-coefficient-of-fraternity}{%
\subsection{The coefficient of fraternity}\label{the-coefficient-of-fraternity}}

\hypertarget{the-genetic-covariance-between-relatives}{%
\section{The genetic covariance between relatives}\label{the-genetic-covariance-between-relatives}}

\hypertarget{the-effect-of-linkage-and-gametic-phase-disequilibrium}{%
\section{The effect of linkage and gametic phase disequilibrium}\label{the-effect-of-linkage-and-gametic-phase-disequilibrium}}

\hypertarget{gametic-phase-disequilibrium}{%
\subsection{Gametic phase disequilibrium}\label{gametic-phase-disequilibrium}}

\hypertarget{assortative-mating}{%
\section{Assortative mating}\label{assortative-mating}}

\hypertarget{polyploidy-1}{%
\section{Polyploidy}\label{polyploidy-1}}

\hypertarget{environmental-sources-of-covariance-between-relatives}{%
\section{Environmental sources of covariance between relatives}\label{environmental-sources-of-covariance-between-relatives}}

\hypertarget{the-heritability-concept}{%
\section{The heritability concept}\label{the-heritability-concept}}

\hypertarget{evolvability}{%
\subsection{Evolvability}\label{evolvability}}

\hypertarget{introduction-to-matrix-algebra-and-linear-models}{%
\chapter{Introduction to Matrix Algebra and Linear Models}\label{introduction-to-matrix-algebra-and-linear-models}}

\[\newcommand{\mx}[1]{\mathbf{#1}}\]

\hypertarget{multiple-regression}{%
\section{Multiple regression}\label{multiple-regression}}

Simple multiple regression equation:

\begin{equation}
    y = \alpha + \beta_1z_1 + \beta_2z_2 + ... + \beta_nz_n + e
    \label{eq:simple-multiple-regression}
\end{equation}

\(y\) = dependent/response variable, \(z_1, z_2, z_n\) = predictors, \(e\) = residual error, \(\alpha\) is a constant as are \(\beta_1, \beta_2, \beta_n\) to be estimated.

Recall from Chapter 3 that the goal of least-squares regression is to find a set of constants (\(\alpha\) and the \(\beta\)s) that minimise the squared differences between observed and expected values, with expected values is anything that fits on the ``line of best fit''. Also, recall equation \eqref{eq:intercept-and-slope}, to see the relationship between \(y\), \(z_n\) (\(x\) in the equation) and \(b\). For multiple regression there are many ``\(b\)'' terms and each of them can be estimated by dividing the covariance of the dependent variable and the predictor (\(\sigma(y, z_n)\)) by the covariance of the predictor with all other predictors in the model. When \(n = 1\), the model reduces to a simple linear regression and we return to equation \eqref{eq:intercept-and-slope}. This can be represented in matrix form like so:

\begin{equation}
    \begin{pmatrix}
        \sigma^2(z_1) & \sigma(z_1, z_n) & \dots & \sigma(z_1, z_n) \\
        \sigma(z_1, z_1) & \sigma^2(z_2) & \dots & \sigma(z_2, z_n) \\
        \vdots & \vdots & \ddots & \vdots \\
        \sigma(z_1, z_n) & \sigma(z_2, z_n) & \dots & \sigma^2(z_n)
    \end{pmatrix}
    \begin{pmatrix}
        \beta_1 \\
        \beta_2 \\
        \vdots \\
        \beta_n
    \end{pmatrix}
    =
    \begin{pmatrix}
        \sigma(y, z_1) \\
        \sigma(y, z_2) \\
        \vdots \\
        \sigma(y, z_n)
    \end{pmatrix}
    \label{eq:multiple-regression-matrix-form}
\end{equation}

When estimating each response variable-predictor covariance term, it is the sum of predictor covariance multiplied by beta.

If the covariance matrix and the vectors of \eqref{eq:multiple-regression-matrix-form} are written as \(\mx{V}\), \(\mx{\beta}\) and \(\mx{c}\) respectively, then the equation can be re-written as:

\begin{equation}
    \mx{V\beta} = \mx{c}
    \label{eq:abbreviated-multiple-regression}
\end{equation}

\textbf{NOTE}: It is standard procedure to denote matrices as bold capital letters and vectors as bold lower case letters.

Before going onto matrix methods in more detail, here is an application of \eqref{eq:simple-multiple-regression} in quantitative genetics.

\hypertarget{an-application-to-multivariate-selection}{%
\subsection{An application to multivariate selection}\label{an-application-to-multivariate-selection}}

Suppose that a large number of individuals in a population have been measured for \(n\) characters and for fitness. Individual fitness can then be approximated by the linear model

\begin{equation}
    w = \alpha + \beta_1z_1 + \beta_2z_2 + ... + \beta_nz_n + e
    \label{eq:fitness-linear-model}
\end{equation}

where \(w\) is the relative fitness (observed fitness divided by the mean fitness in the population). In Chapter 3, we learnt that the selection differential for the \(i\)th trait is defined as the covariance between phenotype and relative fitness, \(S_i = \sigma(z_i, w)\). Therefore, if we use multiple regression to estimate \(S_i\) we'd end up with:

\begin{equation}
    S_i = \beta_i\sigma^2(z_i) + \sum^n_{j \neq i} {\beta_j\sigma(z_i, z_j)}
    \label{eq:ith-selection-differential} 
\end{equation}

Simple!

\hypertarget{elementary-matrix-algebra}{%
\section{Elementary matrix algebra}\label{elementary-matrix-algebra}}

\hypertarget{basic-notation}{%
\subsection{Basic notation}\label{basic-notation}}

Vectors and matrices in mathematics are just like those in R. A matrix with the same number of rows and columns is called a square matrix. Vectors written vertically are called column vectors, e.g.~

\begin{equation}
    a = 
    \begin{pmatrix}
        12 \\
        13 \\
        47
    \end{pmatrix}
    \notag
\end{equation}

and those that are written horizontally are called row vectors, e.g.~

\begin{equation}
    b = 
    \begin{pmatrix}
        12 & 13 & 47
    \end{pmatrix}
    \notag
\end{equation}

Single numbers by themselves are often referred to as scalars.

\hypertarget{questions}{%
\chapter*{Questions}\label{questions}}
\addcontentsline{toc}{chapter}{Questions}

Have fun answering these Gib!

\hypertarget{chapter-4}{%
\section*{Chapter 4}\label{chapter-4}}
\addcontentsline{toc}{section}{Chapter 4}

\begin{enumerate}
\def\labelenumi{\arabic{enumi}.}
\tightlist
\item
  What the fuck are they talking about with the molecular basis of dominance? - page 63-64
\end{enumerate}

\hypertarget{chapter-5}{%
\section*{Chapter 5}\label{chapter-5}}
\addcontentsline{toc}{section}{Chapter 5}

\begin{enumerate}
\def\labelenumi{\arabic{enumi}.}
\tightlist
\item
  How do they calculate the variance of a phenotype explained by just the dominance effects? - page 91
\end{enumerate}

\end{document}

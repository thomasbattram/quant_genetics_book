% Options for packages loaded elsewhere
\PassOptionsToPackage{unicode}{hyperref}
\PassOptionsToPackage{hyphens}{url}
%
\documentclass[
]{book}
\usepackage{lmodern}
\usepackage{amssymb,amsmath}
\usepackage{ifxetex,ifluatex}
\ifnum 0\ifxetex 1\fi\ifluatex 1\fi=0 % if pdftex
  \usepackage[T1]{fontenc}
  \usepackage[utf8]{inputenc}
  \usepackage{textcomp} % provide euro and other symbols
\else % if luatex or xetex
  \usepackage{unicode-math}
  \defaultfontfeatures{Scale=MatchLowercase}
  \defaultfontfeatures[\rmfamily]{Ligatures=TeX,Scale=1}
\fi
% Use upquote if available, for straight quotes in verbatim environments
\IfFileExists{upquote.sty}{\usepackage{upquote}}{}
\IfFileExists{microtype.sty}{% use microtype if available
  \usepackage[]{microtype}
  \UseMicrotypeSet[protrusion]{basicmath} % disable protrusion for tt fonts
}{}
\makeatletter
\@ifundefined{KOMAClassName}{% if non-KOMA class
  \IfFileExists{parskip.sty}{%
    \usepackage{parskip}
  }{% else
    \setlength{\parindent}{0pt}
    \setlength{\parskip}{6pt plus 2pt minus 1pt}}
}{% if KOMA class
  \KOMAoptions{parskip=half}}
\makeatother
\usepackage{xcolor}
\IfFileExists{xurl.sty}{\usepackage{xurl}}{} % add URL line breaks if available
\IfFileExists{bookmark.sty}{\usepackage{bookmark}}{\usepackage{hyperref}}
\hypersetup{
  pdftitle={Notes for: Walsh and Lynch. Genetics and Analysis of Quantitative Traits},
  pdfauthor={Thomas Battram},
  hidelinks,
  pdfcreator={LaTeX via pandoc}}
\urlstyle{same} % disable monospaced font for URLs
\usepackage{longtable,booktabs}
% Correct order of tables after \paragraph or \subparagraph
\usepackage{etoolbox}
\makeatletter
\patchcmd\longtable{\par}{\if@noskipsec\mbox{}\fi\par}{}{}
\makeatother
% Allow footnotes in longtable head/foot
\IfFileExists{footnotehyper.sty}{\usepackage{footnotehyper}}{\usepackage{footnote}}
\makesavenoteenv{longtable}
\usepackage{graphicx}
\makeatletter
\def\maxwidth{\ifdim\Gin@nat@width>\linewidth\linewidth\else\Gin@nat@width\fi}
\def\maxheight{\ifdim\Gin@nat@height>\textheight\textheight\else\Gin@nat@height\fi}
\makeatother
% Scale images if necessary, so that they will not overflow the page
% margins by default, and it is still possible to overwrite the defaults
% using explicit options in \includegraphics[width, height, ...]{}
\setkeys{Gin}{width=\maxwidth,height=\maxheight,keepaspectratio}
% Set default figure placement to htbp
\makeatletter
\def\fps@figure{htbp}
\makeatother
\setlength{\emergencystretch}{3em} % prevent overfull lines
\providecommand{\tightlist}{%
  \setlength{\itemsep}{0pt}\setlength{\parskip}{0pt}}
\setcounter{secnumdepth}{5}
\usepackage{booktabs}
\usepackage{amsthm}
\makeatletter
\def\thm@space@setup{%
  \thm@preskip=8pt plus 2pt minus 4pt
  \thm@postskip=\thm@preskip
}
\makeatother
\usepackage[]{natbib}
\bibliographystyle{apalike}

\title{Notes for: Walsh and Lynch. Genetics and Analysis of Quantitative Traits}
\author{Thomas Battram}
\date{2020-03-28}

\begin{document}
\maketitle

{
\setcounter{tocdepth}{1}
\tableofcontents
}
\hypertarget{preface}{%
\chapter*{Preface}\label{preface}}
\addcontentsline{toc}{chapter}{Preface}

This is a good book, but if I make it through the whole thing I deserve several medals and some cake.

\hypertarget{an-overview-of-quantitative-genetics}{%
\chapter{An overview of quantitative genetics}\label{an-overview-of-quantitative-genetics}}

BORWANG!!!

This chapter just introduces the book and some simple concepts.

\hypertarget{properties-of-distributions}{%
\chapter{Properties of distributions}\label{properties-of-distributions}}

ALSO BORWANG!

You can guess what this chapter was on and also how much of a hoot it was\ldots{}

\hypertarget{covariance-regression-and-correlation}{%
\chapter{Covariance, regression, and correlation}\label{covariance-regression-and-correlation}}

\hypertarget{covariance}{%
\section{Covariance}\label{covariance}}

Covariance is a measure of association and the covariance between x and y would be denoted by \(\sigma(x, y)\). If x and y are independent then \(\sigma(x, y) = 0\), BUT if \(\sigma(x, y) = 0\), x and y aren't necessarily independent.

\hypertarget{useful-identities-for-cov}{%
\subsection{Useful identities for cov}\label{useful-identities-for-cov}}

Covariance of x with itself = variance of x:

\begin{equation}
    \sigma(x, x) = \sigma^2(x)
\end{equation}

For constants (here represented by a):
\[\sigma(a, x) = 0\]
\[\sigma(ax, y) = a\sigma(x, y)\]
\[\sigma^2(a, x) = a^2\sigma^2(x)\]
\[\sigma[(a + x), y] = \sigma(x, y)\]

The covariance of 2 sums can be written as the sum of covariances, i.e.~just multiply out the brackets (I've left this blank, do it yourself or check book):
\[\sigma[(x + y),(w + z)] = ...\]

Variance of a sum is sum of variances and covariances (figure this out):
\[\sigma^2(x + y) = ...\]

\hypertarget{least-squares-linear-regression}{%
\section{Least squares linear regression}\label{least-squares-linear-regression}}

Linear model:
\[y = \alpha + \beta{x} + e\]

Continuing on, \(\alpha\) and \(\beta\) will be the true population values and a and b will be the intercept and slope for the line of best fit derived from observed data. The derivation of a and b using the least-squares model can be found on pages 39-41. Buuut, who cares about that, here are the results:
\[a = \bar{y} - b\bar{x}\]
\[b = \frac{Cov(x, y)} {Var(x)}\]

\hypertarget{properties-of-least-squares}{%
\subsection{Properties of least squares}\label{properties-of-least-squares}}

6 in the book, just writing down important/not obvious ones.

\begin{itemize}
\tightlist
\item
  The mean residual (\(\bar{e}\)) is 0
\item
  Residual errors are uncorrelated with predictor variable x (see book for why)

  \begin{itemize}
  \tightlist
  \item
    BUT e and x may not be independent if the relationship between x and y is non-linear. If it is truly non-linear \(E(e|x) != 0\)
  \end{itemize}
\item
  Variance of e can vary with x, in this situation the the regression is said to display heteroscedasticity (see Figure 3.4 for great illustration)
\item
  The regression of y on x is different to the regression of x on y!
\end{itemize}

\hypertarget{correlation}{%
\section{Correlation}\label{correlation}}

Correlation coefficient between x and y:
\[r(x, y) = \frac{Cov(x, y)} {\sqrt{Var(x) Var(y)}}\]

The correlation coefficient is a dimensionless measure of association and it is symmetrical (i.e.~\(r(x, y) = r(y, x)\)).

Scaling x or y by constants does not change the correlation coefficient, but it does affect variances and covariances.

The correlation coefficient is a standardised regression coefficient -\textgreater{} the regression coefficient resulting from rescaling x and y such that each has unit variance).

\(r^2\) assumes \(E(y|x)\) is linear!

\hypertarget{differential-selection-brief-intro}{%
\section{Differential selection (brief intro)}\label{differential-selection-brief-intro}}

The directional selection differential, \(S\), is the difference between the mean phenotype within that generation before selection (\(\mu_s\)) and the mean phenotype within that generation after (\(\mu\)) selection.
\[S = \mu_s - \mu\]

If all individuals have equal fertility and viability then selecting individuals won't change anything so \(\mu_s = \mu\) and \(S = 0\).

If \(W(z)\) is the probability that individuals with phenotype \(z\) survive to reproduce and \(p(z)\) is the density of \(z\) (pretty much means distribution) before selection, then the density after selection is:
\[p_{s}(z) = \frac{W(z)p(z)} {\int W(z)p(z)dz}\]

The denominator here is the mean individual fitness (\(\bar{W}\)). The relative fitness of \(z\) is \(w(z) = \frac{W(z)} {\bar{W}}\).

After some sweet derivation (see page 46), you finish with:
\[S = \sigma[z, w(z)]\]

Therefore the directional selection is equivalent to the covariance of the phenotype and the relative fitness.

If you regress offspring phenotype on the midparent phenotype and that relationship is linear with slope \(\beta\), a change in mean midparent phenotype induces an expected change in mean phenotype across generations equal to:
\[\Delta\mu = \mu_0 - \mu = \beta(\mu_s - \mu) = \beta{S}\]

This is the breeders' equation!

\hypertarget{correlation-between-genotype-and-phenotype-brief-intro}{%
\section{Correlation between genotype and phenotype (brief intro)}\label{correlation-between-genotype-and-phenotype-brief-intro}}

Only when there is no gene-environment interaction is the variance explained by genetics (broad-sense heritability) the equation below:
\[H^2 = \frac{\sigma^2_G} {\sigma^2_z}\],

where \(z\) is the phenotype and G is the sum of the total effects (not just additive) at all loci on the trait.

The slope of a midparent-offspring regression provides an estimate of the proportion of the phenotypic variance that is attributable to additive genetic factors (the narrow-sense heritability).
\[h^2 = \frac{\sigma^2_A} {\sigma^2_z}\]

So as \(h^2\) is just the regression of offspring phenotype on midparent phenotype it can actually be used in the breeders' equation!
\[\Delta\mu = h^2S\]

So the narrow-sense heritability can be thought of as the efficiency of the response to selection. If \(h^2 = 0\) there can be no evolutionary change regardless of strength of selection. Although this should be obvious because if \(h^2\) is 0 then there is clearly no passing of genetic material onto the next generation that is influencing that trait.

\hypertarget{properties-of-single-loci}{%
\chapter{Properties of single loci}\label{properties-of-single-loci}}

\hypertarget{introduction}{%
\section{Introduction}\label{introduction}}

too easy

\hypertarget{allele-and-genotype-frequencies}{%
\section{Allele and genotype frequencies}\label{allele-and-genotype-frequencies}}

too easy

\hypertarget{the-transmission-of-genetic-information}{%
\section{The transmission of genetic information}\label{the-transmission-of-genetic-information}}

\hypertarget{the-hardy-weinberg-principle}{%
\subsection{The Hardy-Weinberg principle}\label{the-hardy-weinberg-principle}}

\[p^2 + 2pq + q^2 = 1\]

where p = allele frequency of first allele at a locus and q = allele frequency of the second allele at that same locus.

Assumptions of H-W:

\begin{itemize}
\tightlist
\item
  No selection
\item
  No mutation
\item
  Random mating
\item
  No differential migration
\item
  No random drift
\end{itemize}

Even though these assumptions will never be met completely in the real world, for the majority of the time the H-W prinicple holds regardless.

Assuming assumptions are met, 2 important points from H-W:

\begin{enumerate}
\def\labelenumi{\arabic{enumi}.}
\tightlist
\item
  It takes no more than a single generation equilibriate and stabilize gene frequencies in the two sexes.
\item
  Only one additional generation is required for the stabilisation of the genotype frequencies into the predictible Hardy-Weinberg proportions.
\end{enumerate}

\hypertarget{sex-linked-loci}{%
\subsection{Sex-linked loci}\label{sex-linked-loci}}

Alleles on sex chromosomes in diploid organisms are obviously different. Sons can only receive an X chromosome from their mother so the frequency of X linked loci in the sons is equal to that of their mothers. Daughters receive both an X chromosome from Mum + from Dad.

Overall this means allele frequencies oscilate around an equilibrium state, but continually get closer to that state over the generations (see Figure 4.2 and page 56 for equation).

\hypertarget{polyploidy}{%
\subsection{Polyploidy}\label{polyploidy}}

Skipped over this section because it's not relevant to human quant gen. Buuut, essentially it just details how to derive allele frequencies under a certain case of polyploidy. Also, it should be noted that of course H-W does not hold under polyploidy!

\hypertarget{age-structure}{%
\subsection{Age structure}\label{age-structure}}

Age structure also complicates our idealised model of H-W. In populations composed of several age classes, the generations overlap, and this causes the approach of genotype frequencies towards the H-W expectations to be gradual (rather than just by 1/2 generations), even in the case of an autosomal locus. Doesn't explain this very much, but it's covered elsewhere. Importantly, when newly founded populations have significant age structure, fluctuations in both gene and genotoype frequencies may occur for a substantial period of time even in the abscence of selection!

\hypertarget{testing-for-hardy-weinberg-proportions}{%
\subsection{Testing for Hardy-Weinberg proportions}\label{testing-for-hardy-weinberg-proportions}}

Says in the book that LRT can be used to test for departures from HWE, buuuut I'm pretty sure that even now the most common method is the chi-squared test (or Fisher's exact test if the sample size is tiny and the allele is rare.). Essentially, in a population, at a specific locus, you can calculate the allele frequencies (and from that expected genotype frequencies) from the observed genotype frequencies then test if there is a difference between the observed and expected values. LRT equation for it given on page 60. \textbf{RECREATE THE CHI-SQUARED TEST IN CODE!!!!}

Should remember (as pointed out above), that just because some assumptions are violated, doesn't mean you'd get a departure from HWE!

\hypertarget{characterising-the-influence-of-a-locus-on-the-phenotype}{%
\section{Characterising the influence of a locus on the phenotype}\label{characterising-the-influence-of-a-locus-on-the-phenotype}}

If a trait is entirely influenced by a single locus then the genetic effect on that trait can be characterised pretty easily and the dominance and additive effects of the alleles can be calculated. So if a locus has genotypes \(B_1B_1\), \(B_1B_2\), \(B_2B_2\), then the values given to these genotypes can be said to be: \(-a\), \((1 + k)a\) and \(+a\). Now if you have genotype data at that locus and data on the trait you can work out the effect of the \(B_2\) allele by taking the mean phenotypic value of individuals with \(B_2B_2\) and subtracting the mean phenotypic value of individuals with \(B_1B_1\) and dividing by 2 i.e.

\[ B_{2eff} = \frac{p_{B2} - p_{B1}} {2} \]

where \(B_{2eff}\) is the effect of allele \(B_2\), \(p_{B2}\) is the mean phenotypic value of individuals with \(B_2B_2\) and \(p_{B1}\) is the mean phneotypic value of individuals with \(B_1B_1\).

As \(B_{2eff} = a\) you can then substitute this in to \((1 + k)a\) to get the dominance coefficient \(k\). Of course if \(k = 0\) then there is no dominance (in reality you would calculate probability of dominance).

\hypertarget{the-basis-of-dominance}{%
\section{The basis of dominance}\label{the-basis-of-dominance}}

Confusing part\ldots{} Don't really get the enzyme activity bit\ldots{}

Main point (I think) is that new deleterious mutations are very likely to be recessive and new mutations with a slight deleterious effect interact in an almost entirely additive fashion (no dominance!).

\hypertarget{fishers-decomposition-of-the-genotypic-value}{%
\section{Fisher's decomposition of the genotypic value}\label{fishers-decomposition-of-the-genotypic-value}}

Recalling that the phenotypic value can be partitioned like so:

\[ z = G + E \]

where \(z\) is the phenotype, \(G\) is the genotypic value and \(E\) is the environmental value.

The genotypic value of a specific locus can be partitioned into it's ``expected'' values based on there being only additive effects (\(\hat{G}\)) and the deviations from the expected values or dominance effects (\(\delta\)). So for genotype \(B_iB_J\):

\[ G_ij = \hat{G_ij} + \delta_ij \]

This can be formalised (whatever the fuck that means) by regressing the genotypic values on the number of \(B_1\) and \(B_2\) alleles in the genotype (\(N_1\) and \(N_2\)):

\[ G_ij = \hat{G_ij} + \delta_ij = \mu_G + \alpha_1N_1 + \alpha_2N_2 \]

\(\mu_G\) = the mean genotypic value in the population, \(\alpha_1\) and \(\alpha_2\) are the slopes of the regression, \(N_1\) and \(N_2\) are the number of \(B_1\) and \(B_2\) alleles. So the regression is:

\[ G_ij ~ N_2 + N_1 \]

By noting that for any individual, \(N_1 = 2 - N_2\) you can reduce the multiple regression model into an easier to work with univariate model. Give it a go:

\[ G_ij = ... \]

If you plotted the genotypic value (\(G\)) against gene content (\(N_2\) or number of \(B_2\) alleles) and calculated residuals these residuals would be \(\delta\) (see Figure 4.6).

The rest of the chapter uses this regression and what we know about genotype frequencies to derive a formula for the average effect of allelic substitution:

\[ \alpha = a[1 + k(p_1 - p_2)] \]

where \(a\) = genotypic value of \(B_2\) (see above), \(k\) is the dominance coefficient and \(p_1\) and \(p_2\) are the frequencies of \(B_1\) and \(B_2\). This value \(\alpha\) represents the average change in genotypic value that results when a \(B_2\) allele is randomly substituted for a \(B_1\) allele. If no dominance (\(k = 0\)) then \(\alpha = a\). Except in the case of additivity, the average effect of allelic substitution is not simply a function of the inherent physiological properties of the allele. It can only be defined in the context of the population!

\hypertarget{partioning-the-genetic-variance.}{%
\section{Partioning the genetic variance.}\label{partioning-the-genetic-variance.}}

Deriving variance of \(G\):

\[ G = \hat{G} + \delta \]
\[ \sigma^2_G = \sigma^2(\hat{G} + \delta) \]
\[ \sigma^2_G = \sigma^2(\hat{G}) + \sigma(\hat{G} + \delta) + \sigma^2(\delta) \]

The top equation is just like a regression, with \(\delta\) being the residual error and we know that for least-squares there is no correlation between the residual error and the predictor. So there is no correlation between \(\hat{G}\) and \(\delta\). Therefore:

\[ \sigma^2_G = \sigma^2(\hat{G}) + \sigma^2(\delta) \]
OR more commonly
\[ \sigma^2_G = \sigma^2_A + \sigma^2_D \]

\(\sigma^2_A\) is the variance of G explained by regression on \(N_2\) (or \(N_1\)), and \(\sigma^2_D\) is the residual variance of that regression. The variance of the additive and dominance effects!

For a diallelic locus we can do some rearranging of equations in Table 4.1 of the book and get these equations:

\[ \sigma^2_A = 2p_1p_2\alpha^2 \]
\[ \sigma^2_D = (2p_1p_2ak)^2 \]

From these we can clearly see that both components depend on allele frequencies, the dominance coefficient and the homozygous effect (remember \(\alpha\) is just the slope of the \(G ~ N_2\) regression!).

By plotting how genetic variance changes with gene frequency under different scenarios (\textbf{DEFINITELY DO THIS}). You see some interesting patterns. Firstly, at a single diallelic locus, you see that \(\sigma^2_A\) reaches it's peak when \(p_1 = p_2 = 0.5\). Secondly, it's clear that, even in the case of overdominance (which is rare!), additive genetic variance will almost always be much higher than genetic variance from dominance effects, even when the frequency of the dominant allele is high. (btw I think the scale on the top left graph is wrong\ldots).

\hypertarget{additive-effects-average-excesses-and-breeding-values}{%
\section{Additive effects, average excesses and breeding values}\label{additive-effects-average-excesses-and-breeding-values}}

The dominance deviation of a parent, which is a function of the interaction between the two parental alleles, is eliminated when gametes are produced. Thus, one can think of \(\hat{G}\) and \(\delta\) as the heritable and nonheritable components of an individual's genotypic value.

Fisher proposed two different measures of the effect of an allele: one being the additive effect (\(\alpha_i\)) and then the average excess \(\alpha^x_i\). The average excess \(\alpha^x_2\) of allele \(B_2\) is the difference between the mean genotypic value of individuals carrying at least one copy of \(B_2\) and the mean genotypic value of a random individual from the entire population:

\[ \alpha^x_2 = (G_{12}P_{12|2} + G_{22}P_{22|2}) - \mu_G \]

where \(P_{ij}\) is the conditional probability of a \(B_iB_j\) genotype given that one allele is \(B_i\). Under random mating \(P_{ij|i} = p_j\) (\(p_j\) = frequency of allele \(B_j\)). THINK ABOUT HARDY-WEINBERG AND IT MAKES SENSE!

So under random mating,

\[ \alpha^x_2 = G_{12}p_1 + G_{22}p_2 - \mu_G \]

\(G_{12} = a(1+k)\) and \(G_{22} = 2a\). By substituting these into the equation above for \(\alpha^x_1\) and \(\alpha^x_2\) and then calculating \(\alpha_1\) and \(\alpha_2\) (additive effects) by the method previously mentioned (regressing genotypic value \(G\) on the number of \(B_2\) alleles \(N_2\)), we will see they're equivalent (shown on page 72):

\[ \alpha_2 = p_1\alpha \]
\[ \alpha_1 = -p_2\alpha \]

The breeding value of an individual (\(A\)) is the sum of the additive effects of its genes. So the breeding value of a \(B_1B_1\) homozygote is just \(2\alpha_1\). In randomly mating populations the breeding value of a genotype is equivalent to twice the expected deviation of its offspring mean phenotype from the population mean. Soooo, no genotype information is needed to calculate the breeding value. All we have to do is mate an individual to many randomly chosen individuals from the population and taking twice the deviation of its offspring mean from the population mean. EASY IN HUMANS!!!

In Chapter 13 this will be discussed wrt candidate gene studies.

\hypertarget{extensions-for-multiple-alleles-and-non-random-mating}{%
\section{Extensions for multiple alleles and non random mating}\label{extensions-for-multiple-alleles-and-non-random-mating}}

So this section seems mostly unrelevant as we're unlikely to deal with situations with more than 2 alleles. Non-random mating could be encountered if we're interested in some phenotypes (e.g.~alcohol intake). Buuuut, it's still good to note some of the generalised equations for what we've been discussing so far in the chapter.

\hypertarget{average-excess}{%
\subsection{Average excess}\label{average-excess}}

When \(n\) alleles are present, the average excess, \(\alpha^x_i\), for any allele \(B_i\) is given by

\[ \alpha^x_i = \sum_{j=1}^{n} P_{ij|i}G_{ij} - \mu_G \]

Remember, under random mating \(P_{ij|i} == p_j\)

\hypertarget{additive-effects}{%
\subsection{Additive effects}\label{additive-effects}}

The genotypic value can also be obtained using regression as before, but in it's generalised form is a multivariate regression. For \(n\) alleles

\[ G = \mu_G + \sum_{i=1}^{n} \alpha_{i}N_{i} + \delta \]

After some re-arranging can derive the regression coefficients and finally end with

\[ \alpha_i = \sum_{j=1}^{n} p_jG_{ij} - \mu_G \]

i.e.~under random mating, the average effects (\(\alpha_i\)) are equal to the conditional mean deviations from the mean genotypic value of the population (\(\mu_G\)).

For non-random mating we need the inbreeding coefficient, \(f\) to define our genotype frequencies:

\[ P_{ii} = (1 - f)p^2_i + fp_i \]
\[ P_{ij} = 2(1 -f)p_ip_j \]

Unsure of why, but this means

\[ \alpha_i = \frac{\alpha^x_i} {1 + f} \]

so \(f\) is the fractional reduction of heterozygote frequencies relative to those expected under random mating. This means you can kind of do a test for random mating by checking heterozygote and homozygote frequencies in a population!

\hypertarget{additive-genetic-variance}{%
\subsection{Additive genetic variance}\label{additive-genetic-variance}}

The additive genetic variance across \(n\) alleles is

\[ \sigma^2_A = 2 \sum_{i=1}^{n} p_i\alpha_i\alpha^x_i \]

In general inbreeding inflates the additive genetic variance by causing correlations among the effects of alleles within the same individuals.

The broad sense heritability, even under scenarios of non-random mating can be given by

\[ \sigma^2_G = \sigma^{2}(\alpha_i + \alpha_j) + \sigma^{2}(\delta_{ij}) \]

although it should be noted that the definitions of \(\alpha_i\) and \(\delta_ij\) change with the degree of inbreeding! Random mating means \(\alpha_i\) and \(\alpha_j\) are uncorrelated so we get back to the good old equation

\[ \sigma^2_G = \sigma^2_A + \sigma^2_D \]

Importantly, under random mating, \(\sigma^2_A\) is equivalent to the variance of breeding values of individuals in the population.

\begin{center}\rule{0.5\linewidth}{0.5pt}\end{center}

\textbf{Summarising some key terms}

\textbf{The homozygous effect, \(a\), and the dominance coefficient, \(k\),} are intrinsic properties of allelic products. They are not functions of allele frequencies, but may vary with genetic background

\textbf{The additive effect, \(\alpha_i\), and the average excess, \(\alpha^x_i\),} are properties of alleles in a particular population. They are functions of \(a\), \(k\) and genotype frequencies (\(p_i\)).

\textbf{The breeding value, \(A\),} is a property of a particular individual in reference to a particular population. It's equivalent to the sum of the additive effects of an individual's alleles.

\textbf{The additive genetic variance, \(\sigma^2_A\)} is a property of a particular population. It is equivalent to the variance of the breeding values of individuals within the population.

\begin{center}\rule{0.5\linewidth}{0.5pt}\end{center}

  \bibliography{book.bib,packages.bib}

\end{document}
